\documentclass[10pt,landscape]{article}
\usepackage[utf8]{inputenc}
\usepackage[margin=0.4in]{geometry}
\usepackage{amsmath, amssymb}
\usepackage{multicol}
\usepackage[breakable]{tcolorbox}
\usepackage{enumitem}

\pagestyle{empty}
\setlength{\columnsep}{0.2in}
\renewcommand{\arraystretch}{1.1}
\begin{document}
\footnotesize

% --------------------- Page 1 ---------------------
\begin{multicols}{3}

%%%%%%%%%%%%%%%%%%%%%%%%%%%%%%%%%%%%%%%%%%
\begin{tcolorbox}[breakable, title=Common Patterns Across Models]
\begin{enumerate}
    \item \textbf{Modeling:} Decide whether to model \(p(y\mid x)\) (discriminative) or \(p(x,y)\) (generative).\\[1mm]
    \item \textbf{Loss Functions:} Examples include squared error, cross-entropy, hinge loss, etc.\\[1mm]
    \item \textbf{Optimization:} Solve in closed form (e.g. linear regression) or use iterative methods such as gradient descent/backpropagation.\\[1mm] 
\end{enumerate}

\textbf{Discriminative Models:} Directly model
\[
p(y\mid x;w).
\]
For binary logistic regression:
\[
p(y=1\mid x;w)=\sigma(w^T x),\quad \sigma(z)=\frac{1}{1+e^{-z}}.
\]
\textbf{Generative Models:} Model \(p(x,y)\) then apply Bayes' rule to compute \(p(y\mid x)\).\\[1mm]
\textbf{MLE \& MAP:}\\
\[
\begin{aligned}
w_{\mathrm{MLE}} &= \arg\max_w \log\, p((x,y)\mid w),\\[1mm]
w_{\mathrm{MAP}} &= \arg\max_w \Bigl\{\log\, p((x,y)\mid w)+\log\, p(w)\Bigr\},
\end{aligned}
\]
with a Gaussian prior \(w\sim\mathcal{N}(0,\sigma_0^2 I)\) yielding ridge regression:
\[
w_{\mathrm{MAP}}=\arg\min_w \sum_{i=1}^N \left(y^{(i)}-w^T x^{(i)}\right)^2+\lambda\|w\|_2^2,\quad \lambda=\frac{\sigma^2}{\sigma_0^2}.
\]
\end{tcolorbox}

%%%%%%%%%%%%%%%%%%%%%%%%%%%%%%%%%%%%%%%%%%
\begin{tcolorbox}[breakable, title=Optimal \(w\) \& Likelihood Functions]
\textbf{Linear Regression:}
\[
w^*=(X^T X)^{-1}X^Ty,\quad p((x,y)\mid w)=\prod_{i=1}^N \mathcal{N}\bigl(y^{(i)}\mid w^T x^{(i)},\,\sigma^2\bigr).
\]
\textbf{Binary Logistic Regression:}
\[
p(y\mid x;w)=\sigma(w^T x)^y\,[1-\sigma(w^T x)]^{1-y},
\]
with log-likelihood
\[
\ell(w)=\sum_{i=1}^N \Bigl[y^{(i)}\log\sigma(w^T x^{(i)})+(1-y^{(i)})\log(1-\sigma(w^T x^{(i)}))\Bigr].
\]
\textbf{Multiclass (Softmax):}\\[1mm]
\textit{Dataset Likelihood:}
\[
\mathcal{L}(\{w_\ell\}) = \prod_{i=1}^N \prod_{k=1}^K \left(\frac{\exp\bigl(w_k^T x^{(i)}\bigr)}{\sum_{\ell=1}^K \exp\bigl(w_\ell^T x^{(i)}\bigr)}\right)^{\mathbb{I}\{y^{(i)}=k\}}.
\]
Taking the logarithm gives the log-likelihood:
\[
\ell(\{w_\ell\}) = \sum_{i=1}^N \left[ w_{y^{(i)}}^T x^{(i)} - \log\left(\sum_{\ell=1}^K \exp\bigl(w_\ell^T x^{(i)}\bigr)\right) \right].
\]
\textbf{Ridge Regression:}
\[
w^*=(X^TX+\lambda I)^{-1}X^Ty.
\]
\end{tcolorbox}

%%%%%%%%%%%%%%%%%%%%%%%%%%%%%%%%%%%%%%%%%%
\begin{tcolorbox}[breakable, title=Loss Functions]
\begin{itemize}[noitemsep, topsep=0pt]
  \item \textbf{0/1 Loss:} \(L(y,\hat{y})=\mathbb{I}(y\neq\hat{y})\).
  \item \textbf{Hinge Loss:} \(L(y,f(x))=\max(0,\,1-y\,f(x))\).
  \item \textbf{L1 Loss:} \(L(y,\hat{y})=|y-\hat{y}|\).
  \item \textbf{L2 Loss:} \(L(y,\hat{y})=(y-\hat{y})^2\).
  \item \textbf{Binary Cross-Entropy:} \(L(y,\hat{y})=-\Bigl[y\log\hat{y}+(1-y)\log(1-\hat{y})\Bigr]\).
  \item \textbf{Softmax Loss:} \(L=-\log\frac{\exp(w_k^T x)}{\sum_\ell \exp(w_\ell^T x)}\).
\end{itemize}
\end{tcolorbox}

%%%%%%%%%%%%%%%%%%%%%%%%%%%%%%%%%%%%%%%%%%
\begin{tcolorbox}[breakable, title=Gradient Descent]
\textbf{Gradient Descent:}
\begin{itemize}[noitemsep, topsep=0pt]
  \item Iteratively update parameters by moving opposite to the gradient:
  \[
  w \leftarrow w - \eta \nabla L(w),
  \]
  where \(\eta\) is the learning rate.
  \item A proper choice of \(\eta\) is crucial: if too high, updates overshoot minima; if too low, convergence is slow.
\end{itemize}
\textbf{Stochastic Gradient Descent (SGD):}
\begin{itemize}[noitemsep, topsep=0pt]
  \item Approximates the full gradient using a single (or a mini-batch of) training example(s):
  \[
  w \leftarrow w - \eta \nabla L^{(i)}(w),
  \]
  where \(L^{(i)}(w)\) is the loss for the \(i\)th example.
  \item Benefits: faster iterations and potential to escape shallow local minima.
  \item Trade-off: introduces variance in updates, often requiring a decaying learning rate schedule.
\end{itemize}
\end{tcolorbox}

%%%%%%%%%%%%%%%%%%%%%%%%%%%%%%%%%%%%%%%%%%
\begin{tcolorbox}[breakable, title=Matrix Rules]
\textbf{Algebra:}
\[
(AB)^T=B^TA^T,\quad (A^{-1})^T=(A^T)^{-1}.
\]
For column vectors \(a,b\): \(a^T b\) is scalar. If \(X\) is \(n\times d\) and \(w\) is \(d\times1\), then \(Xw\) is \(n\times1\).\\[1mm]
\textbf{Derivatives:}
\[
\frac{\partial}{\partial w}(w^TAw)=(A+A^T)w\quad (\text{or }2Aw \text{ if }A\text{ is symmetric}),
\]
\[
\frac{\partial}{\partial w}\frac{1}{2}\|y-Xw\|_2^2=-X^T(y-Xw).
\]
\textbf{Norms:}
\[
\|w\|_2=\sqrt{w^Tw},\quad \|w\|_1=\sum_i|w_i|.
\]
\end{tcolorbox}

%%%%%%%%%%%%%%%%%%%%%%%%%%%%%%%%%%%%%%%%%%
\begin{tcolorbox}[breakable, title=Lagrangian Method]
\textbf{Steps:}
\begin{enumerate}[noitemsep, topsep=0pt]
  \item Form the Lagrangian: \(\mathcal{L}(w,\lambda)=\text{Objective}(w)+\lambda\,(\text{Constraint}(w))\).
  \item Differentiate with respect to \(w\) and \(\lambda\).
  \item Set derivatives to zero and solve.
\end{enumerate}
\textbf{Example (SVM):}
\[
\mathcal{L}(w,b,\lambda)=\frac{1}{2}\|w\|_2^2-\sum_{i=1}^N\lambda_i\Bigl[y^{(i)}(w^Tx^{(i)}+b)-1\Bigr].
\]
\end{tcolorbox}

%%%%%%%%%%%%%%%%%%%%%%%%%%%%%%%%%%%%%%%%%%
\begin{tcolorbox}[breakable, title=Terminology \& Notation]
\textbf{Terminology:}\\[1mm]
\textbf{Posterior:} \(p(w\mid(x,y))\) after observing data.\\[1mm]
\textbf{Posterior Predictive:}
\[
p(y^*\mid x^*,(x,y))=\int p(y^*\mid x^*,w)\,p(w\mid(x,y))\,dw.
\]
\textbf{Marginal Likelihood:}
\[
p((x,y))=\int p((x,y)\mid w)\,p(w)\,dw.
\]
\textbf{Class-Conditional:} \(p(x\mid y)\).\\[2mm]
\textbf{Notation:}\\[1mm]
\begin{itemize}[noitemsep, topsep=0pt]
    \item \(N\): Number of training examples.
    \item \(K\): Number of classes.
    \item \(x^{(i)}\): \(i\)th input data point.
    \item \(y^{(i)}\): Label corresponding to \(x^{(i)}\).
    \item \(X\): Design matrix whose rows are \(x^{(i)}\).
    \item \(w\), \(w_\ell\): Weight vector(s); \(w_\ell\) denotes the weight for class \(\ell\) in multiclass models.
    \item \(w_0\) or \(b\): Bias term.
    \item \(\eta\): Learning rate in gradient descent.
    \item \(\lambda\): Regularization parameter (ridge: \(\ell_2\), lasso: \(\ell_1\)).
    \item \(C\): SVM regularization parameter trading off margin and classification errors.
    \item \(\xi_i\): Slack variable for the \(i\)th example in soft-margin SVM.
    \item \(\phi(x)\): Feature mapping or basis function.
    \item \(\sigma(z)\): Sigmoid function, \(\frac{1}{1+e^{-z}}\).
    \item \(\mathbb{I}\{\cdot\}\): Indicator function.
    \item \(L\): Generic loss function.
    \item \(\ell\): Often denotes log-likelihood.
\end{itemize}
\end{tcolorbox}

\end{multicols}

% --------------------- Page 2 ---------------------
\newpage
\begin{multicols}{3}

%%%%%%%%%%%%%%%%%%%%%%%%%%%%%%%%%%%%%%%%%%
\begin{tcolorbox}[breakable, title=Bias--Variance \& Regularization]
\textbf{Bias--Variance Tradeoff:} High bias \(\rightarrow\) underfitting; high variance \(\rightarrow\) overfitting. Regularization (e.g. ridge, lasso) can reduce variance.\\[2mm]
\textbf{Ridge Regression:}\\
\[
w^*=(X^TX+\lambda I)^{-1}X^Ty.
\]
\textbf{Lasso Regression:}\\
Minimize
\[
\sum_{i=1}^N \left(y^{(i)}-w^T x^{(i)}\right)^2+\lambda\|w\|_1.
\]
\end{tcolorbox}

%%%%%%%%%%%%%%%%%%%%%%%%%%%%%%%%%%%%%%%%%%
\begin{tcolorbox}[breakable, title=Neural Networks]
\textbf{Architecture:}\\
\(\bullet\) Feedforward NN with one hidden layer: 
\[
h = \sigma(W_1 x + b_1),\quad f = W_2 h + b_2.
\]
\(\bullet\) Deep networks with multiple hidden layers enable hierarchical feature learning.\\[1mm]
\textbf{Activation Functions:}\\
Introduce non-linearity (e.g. Sigmoid, ReLU, tanh) to enable universal approximation.\\[1mm]
\textbf{Backpropagation:}\\
Compute gradients via the chain rule:
\[
\frac{\partial L}{\partial W} = \frac{\partial L}{\partial f}\frac{\partial f}{\partial W}.
\]
Performs a forward pass to compute outputs and a backward pass to update parameters.\\[1mm]
\textbf{Additional Points:}
\begin{itemize}[noitemsep, topsep=0pt]
    \item \textbf{Model Selection:} Techniques such as cross-validation and regularization are key for avoiding overfitting.
    \item \textbf{Loss Functions:} Choice depends on the task—least squares for regression; softmax loss for classification.
    \item \textbf{Task Adaptation:} Neural networks can be tailored for both regression and classification tasks.
    \item \textbf{Bias Inclusion:} Always incorporate the bias term via the augmented input (see Supervised Learning Organization).
\end{itemize}
\end{tcolorbox}

%%%%%%%%%%%%%%%%%%%%%%%%%%%%%%%%%%%%%%%%%%
\begin{tcolorbox}[breakable, title=Support Vector Machines (SVMs)]
\textbf{Maximum Margin Classifier:}\\
Finds the hyperplane that maximizes the distance (margin) between classes.\\[1mm]
\textbf{Hard Margin SVM:}\\
\[
\min_{w,b} \frac{1}{2}\|w\|_2^2 \quad \text{s.t. } y^{(i)}(w^T x^{(i)}+b) \ge 1.
\]
\textbf{Soft Margin SVM:}\\
\[
\min_{w,b} \frac{1}{2}\|w\|_2^2 + C\sum_{i=1}^N \xi_i \quad \text{s.t. } y^{(i)}(w^T x^{(i)}+b) \ge 1-\xi_i,\quad \xi_i\ge0.
\]
\textbf{Additional Points:}
\begin{itemize}[noitemsep, topsep=0pt]
    \item \textbf{Hinge Loss:} Penalizes points within the margin, defined as \(L(y, f(x))=\max(0,\,1-y\,f(x))\).
    \item \textbf{Regularization (\(C\)):} A higher \(C\) emphasizes minimizing classification errors, potentially at the cost of a smaller margin.
    \item \textbf{Kernel Trick:} Replaces inner products \(x^Tz\) with \(K(x,z)=\phi(x)^T\phi(z)\) to handle nonlinearly separable data. Common kernels include linear, polynomial, and RBF.
    \item \textbf{Example Kernel:} For RBF, \(K(x,x')=\exp(-\gamma\|x-x'\|_2^2)\).
    \item \textbf{Support Vectors:} The data points that lie closest to the decision boundary; they determine the position of the hyperplane.
    \item \textbf{Decision Boundaries:} SVMs often yield sharper boundaries compared to logistic regression.
\end{itemize}
\end{tcolorbox}

%%%%%%%%%%%%%%%%%%%%%%%%%%%%%%%%%%%%%%%%%%
\begin{tcolorbox}[breakable, title=Naive Bayes \& Bayesian Linear Regression]
\textbf{Naive Bayes:}\\[1mm]
\(\bullet\) \textbf{Modeling:} A generative model that estimates \(p(x,y)\) by assuming feature independence given the class:
\[
p(x\mid y) = \prod_{j} p(x_j \mid y).
\]
\(\bullet\) \textbf{Classification:} Compute the posterior via Bayes' rule:
\[
p(y\mid x) \propto p(y)\,p(x\mid y).
\]
\(\bullet\) \textbf{Characteristics:} Simple, fast, and effective in high-dimensional settings, though it relies on the independence assumption.

\vspace{1mm}
\textbf{Working with Generative Models for Classification:}\\[1mm]
\(\bullet\) Estimate class priors \(p(y)\) and class-conditional likelihoods \(p(x\mid y)\) from the data.\\
\(\bullet\) Apply Bayes' rule to obtain \(p(y\mid x)\) and classify by choosing the class with maximum posterior probability.

\vspace{1mm}
\textbf{Discriminative vs. Generative Models:}\\[1mm]
\(\bullet\) \textbf{Discriminative models} (e.g., Logistic Regression) directly model \(p(y\mid x)\) focusing on the decision boundary.\\
\(\bullet\) \textbf{Generative models} (e.g., Naive Bayes) model the joint distribution \(p(x,y)\) and derive \(p(y\mid x)\) using Bayes' rule.\\
\(\bullet\) Discriminative models often achieve higher asymptotic accuracy, while generative models can perform better with limited data or when model assumptions hold.

\vspace{1mm}
\textbf{Bayesian Linear Regression:}\\[1mm]
\(\bullet\) \textbf{Concept:} Treats weights \(w\) as random variables with a prior (commonly Gaussian).\\
\(\bullet\) \textbf{Posterior:} Update beliefs with:
\[
p(w\mid D) \propto p(D\mid w)\,p(w).
\]
\(\bullet\) \textbf{Prediction:} Integrate over the posterior to obtain:
\[
p(y^*\mid x^*,D)=\int p(y^*\mid x^*,w)\,p(w\mid D)\,dw.
\]
\end{tcolorbox}

\end{multicols}

\end{document}
