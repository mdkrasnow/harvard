\documentclass{article}
\usepackage{amsmath, amssymb}
\usepackage{hyperref}
\usepackage{enumitem}
\hypersetup{
  colorlinks=true,
  linkcolor=blue,
  urlcolor=blue,
}
\begin{document}

\title{Foundations of Real Analysis\\
\large Perfect Sets, Connected Sets, and Numerical Sequences and Series}
\author{}
\date{}
\maketitle

\tableofcontents

\bigskip

\section{Perfect Sets} \label{perfect-sets}

\subsection{Definition and Properties} \label{perfect-sets-definition}

A set $P$ in a topological space is said to be \textbf{perfect} if it is closed and has no isolated points. This means:
\begin{itemize}
    \item $P$ is closed (contains all its limit points)
    \item Every point in $P$ is a limit point of $P$
\end{itemize}

In more formal terms, a point $x$ is a limit point of a set $P$ if every neighborhood of $x$ contains at least one point of $P$ different from $x$ itself.

\textbf{Key Properties of Perfect Sets:}
\begin{enumerate}[label=(\arabic*)]
    \item Every perfect set is closed by definition.
    \item Perfect sets are uncountable (when non-empty in $\mathbb{R}^n$).
    \item The union of finitely many perfect sets is perfect.
    \item The intersection of any collection of perfect sets is perfect.
\end{enumerate}

\subsection{Examples of Perfect Sets} \label{perfect-sets-examples}

\begin{enumerate}[label=\textbf{\arabic*.}]
    \item \textbf{The Real Line $\mathbb{R}$}: \\
    $\mathbb{R}$ is perfect because it is closed (it contains all its limit points) and has no isolated points (every point has other points arbitrarily close to it).
    
    \item \textbf{Closed Intervals $[a,b]$ in $\mathbb{R}$}: \\
    Any closed interval is perfect for the same reasons as $\mathbb{R}$.
    
    \item \textbf{The Unit Circle $S^1$ in $\mathbb{R}^2$}: \\
    The set $\{(x,y) \in \mathbb{R}^2 : x^2 + y^2 = 1\}$ is perfect.
    
    \item \textbf{Empty Set $\emptyset$}: \\
    Vacuously perfect as it has no points, hence no isolated points.
\end{enumerate}

\subsection{The Cantor Set: A Perfect Nowhere Dense Set} \label{cantor-set}

The Cantor set is perhaps the most fascinating example of a perfect set. It's constructed by repeatedly removing the middle third of intervals:

\textbf{Construction:}
\begin{enumerate}[label=\textbf{\arabic*.}]
    \item Start with the interval $C_0 = [0,1]$.
    \item Remove the middle third to get 
    \[
    C_1 = \left[0,\frac{1}{3}\right] \cup \left[\frac{2}{3},1\right].
    \]
    \item Remove the middle third of each remaining interval to get 
    \[
    C_2 = \left[0,\frac{1}{9}\right] \cup \left[\frac{2}{9},\frac{1}{3}\right] \cup \left[\frac{2}{3},\frac{7}{9}\right] \cup \left[\frac{8}{9},1\right].
    \]
    \item Continue this process infinitely.
\end{enumerate}

The Cantor set 
\[
C = \bigcap_{n=0}^{\infty} C_n
\]
has remarkable properties:
\begin{itemize}
    \item It is uncountable (has the same cardinality as $\mathbb{R}$)
    \item It has Lebesgue measure zero (its "length" is zero)
    \item It is perfect (closed with no isolated points)
    \item It is nowhere dense (its interior is empty)
\end{itemize}

\textbf{Real-World Connection:} The Cantor set appears in signal processing as a model for certain types of noise and in the theory of fractals. Its self-similar structure makes it important in studying chaotic dynamical systems.

\subsection{Applications of Perfect Sets} \label{perfect-sets-applications}

\begin{enumerate}[label=\textbf{\arabic*.}]
    \item \textbf{Mathematical Analysis}: Perfect sets provide examples that test the limits of our intuition about continuity and measure.
    \item \textbf{Fractals}: Many fractals are perfect sets, and the study of perfect sets helps us understand the geometry of fractals.
    \item \textbf{Dynamical Systems}: The behavior of certain dynamical systems can be understood by studying perfect sets in their phase spaces.
    \item \textbf{Digital Signal Processing}: Understanding perfect sets helps in analyzing certain types of signals and noise patterns.
    \item \textbf{Descriptive Set Theory}: Perfect sets play a fundamental role in the classification of sets.
\end{enumerate}

\textbf{Example from Economics:} \\
In economic modeling, perfect sets can represent equilibrium states in market systems where small perturbations lead to other nearby equilibrium states, creating a ``perfect'' set of equilibria.

\bigskip

\section{Connected Sets} \label{connected-sets}

\subsection{Definition and Properties} \label{connected-sets-definition}

A set $S$ in a topological space is \textbf{connected} if it cannot be represented as the union of two disjoint, non-empty open sets. Equivalently, $S$ is connected if the only sets that are both open and closed in the subspace topology of $S$ are $\emptyset$ and $S$ itself.

\textbf{Key Properties:}
\begin{enumerate}[label=(\arabic*)]
    \item A set $S$ is connected if and only if there do not exist two disjoint non-empty open sets $U$ and $V$ such that $S \subset U \cup V$ and $S \cap U \neq \emptyset$, $S \cap V \neq \emptyset$.
    \item Connectedness is preserved under continuous functions.
    \item The closure of a connected set is connected.
    \item The union of connected sets that have a point in common is connected.
\end{enumerate}

\subsection{Connectedness in Different Spaces} \label{connectedness-spaces}

\begin{enumerate}[label=\textbf{\arabic*.}]
    \item \textbf{In $\mathbb{R}$}: \\
    A subset of $\mathbb{R}$ is connected if and only if it is an interval.
    
    \item \textbf{In $\mathbb{R}^2$}: \\
    Connected sets in the plane can have much more complex shapes than intervals. For example, star-shaped regions and simply connected domains are connected.
    
    \item \textbf{Discrete Spaces}: \\
    In a discrete topological space, the only connected sets are singletons (sets containing just one point).
\end{enumerate}

\textbf{Real-World Example - Electrical Networks:} \\
The concept of connectedness appears naturally in electrical networks. A circuit is connected if there is a path of conductors between any two points. This ensures that electricity can flow throughout the entire network.

\subsection{Path-Connected Sets} \label{path-connected}

A stronger notion of connectedness is \textbf{path-connectedness}. A set $S$ is path-connected if for any two points $a, b \in S$, there exists a continuous function $f: [0,1] \to S$ such that $f(0) = a$ and $f(1) = b$.

\textbf{Properties:}
\begin{enumerate}[label=(\arabic*)]
    \item Every path-connected set is connected.
    \item The converse is not always true: there exist connected sets that are not path-connected.
    \item In most ``nice'' spaces, connectedness and path-connectedness coincide.
\end{enumerate}

\textbf{Example of a Connected but Not Path-Connected Set:} \\
The ``Topologist's Sine Curve'' defined as:
\[
S = \{(x, \sin(1/x)) : x \in (0,1]\} \cup \{(0,y) : -1 \leq y \leq 1\}
\]
This set is connected but not path-connected because there's no continuous path from any point on the vertical line segment to any point on the oscillating curve.

\subsection{Components and Local Connectedness} \label{components}

The \textbf{connected components} of a set are its maximal connected subsets.

A space is \textbf{locally connected} at a point $p$ if every neighborhood of $p$ contains a connected neighborhood of $p$. A space is locally connected if it is locally connected at each of its points.

\textbf{Real-World Application - Image Processing:} \\
In image processing, connected component analysis identifies regions of connected pixels in binary images. This is crucial for object recognition and feature extraction.

\subsection{Applications of Connected Sets} \label{connected-sets-applications}

\begin{enumerate}[label=\textbf{\arabic*.}]
    \item \textbf{Circuit Design}: Ensuring connectedness in circuit layouts guarantees electrical flow.
    \item \textbf{Network Theory}: Connectedness measures how robustly a network can maintain communication between nodes.
    \item \textbf{Geographic Information Systems (GIS)}: Connected regions represent coherent geographic entities.
    \item \textbf{Computer Graphics}: Path-connected regions are essential for defining objects that can be rendered continuously.
    \item \textbf{Robotics}: Path planning relies on understanding the connectedness of configuration spaces.
\end{enumerate}

\textbf{Example from Urban Planning:} \\
Urban planners use connectedness to analyze transportation networks. A well-connected road system ensures that people can travel between any two points in a city, which is modeled mathematically as a connected graph.

\bigskip

\section{Numerical Sequences and Series} \label{sequences-series}

\subsection{Sequences: Definition and Convergence} \label{sequences-definition}

A \textbf{sequence} is a function whose domain is the set of natural numbers. We denote a sequence as $\{a_n\}_{n=1}^{\infty}$ or simply $\{a_n\}$.

A sequence $\{a_n\}$ \textbf{converges} to a limit $L$ if for every $\epsilon > 0$, there exists an $N \in \mathbb{N}$ such that for all $n \geq N$, 
\[
|a_n - L| < \epsilon.
\]
We write 
\[
\lim_{n \to \infty} a_n = L \quad \text{or} \quad a_n \to L \text{ as } n \to \infty.
\]

If a sequence does not converge, we say it \textbf{diverges}.

\textbf{Examples of Sequences:}
\begin{enumerate}[label=\textbf{\arabic*.}]
    \item $a_n = \dfrac{n}{n+1}$: Converges to 1.
    \item $a_n = (-1)^n$: Diverges (oscillates between -1 and 1).
    \item $a_n = n$: Diverges to infinity.
    \item $a_n = \dfrac{1}{n}$: Converges to 0.
\end{enumerate}

\textbf{Real-World Example - Compound Interest:} \\
If you invest $P$ dollars at an annual interest rate $r$ compounded $n$ times per year, the amount after $t$ years is:
\[
A = P\left(1 + \frac{r}{n}\right)^{nt}.
\]
As $n \to \infty$ (continuous compounding), this sequence converges to 
\[
A = Pe^{rt}.
\]

\subsection{Limit Theorems for Sequences} \label{limit-theorems}

If $\{a_n\}$ and $\{b_n\}$ are convergent sequences with $\lim_{n \to \infty} a_n = L$ and $\lim_{n \to \infty} b_n = M$, then:
\begin{enumerate}[label=\textbf{\arabic*.}]
    \item \textbf{Sum Rule}: 
    \[
    \lim_{n \to \infty} (a_n + b_n) = L + M.
    \]
    \item \textbf{Difference Rule}: 
    \[
    \lim_{n \to \infty} (a_n - b_n) = L - M.
    \]
    \item \textbf{Product Rule}: 
    \[
    \lim_{n \to \infty} (a_n \cdot b_n) = L \cdot M.
    \]
    \item \textbf{Quotient Rule}: 
    \[
    \lim_{n \to \infty} \frac{a_n}{b_n} = \frac{L}{M} \quad \text{if } M \neq 0.
    \]
    \item \textbf{Constant Multiple Rule}: 
    \[
    \lim_{n \to \infty} (c \cdot a_n) = c \cdot L \quad \text{for any constant } c.
    \]
\end{enumerate}

\textbf{The Squeeze Theorem:} \\
If $a_n \leq b_n \leq c_n$ for all $n \geq N$, and 
\[
\lim_{n \to \infty} a_n = \lim_{n \to \infty} c_n = L,
\]
then 
\[
\lim_{n \to \infty} b_n = L.
\]

\subsection{Monotone Sequences and Bounded Sequences} \label{monotone-sequences}

A sequence $\{a_n\}$ is \textbf{monotone increasing} if $a_n \leq a_{n+1}$ for all $n \in \mathbb{N}$, and \textbf{monotone decreasing} if $a_n \geq a_{n+1}$ for all $n \in \mathbb{N}$.

A sequence $\{a_n\}$ is \textbf{bounded} if there exist $M, m \in \mathbb{R}$ such that 
\[
m \leq a_n \leq M \quad \text{for all } n \in \mathbb{N}.
\]

\textbf{Monotone Convergence Theorem:} \\
Every bounded monotone sequence converges.

\textbf{Examples:}
\begin{enumerate}[label=\textbf{\arabic*.}]
    \item $a_n = 1 - \dfrac{1}{n}$: Monotone increasing and bounded above by 1, so it converges to 1.
    \item $a_n = \dfrac{n+1}{n}$: Monotone decreasing and bounded below by 1, so it converges to 1.
    \item $a_n = n$: Monotone increasing but unbounded, so it diverges.
\end{enumerate}

\textbf{Real-World Example - Drug Concentration:} \\
When a patient takes a medication regularly, the concentration in their bloodstream forms a sequence. With repeated doses, this sequence approaches a steady state (converges) if the drug is eliminated at a constant rate.

\subsection{Cauchy Sequences} \label{cauchy-sequences}

A sequence $\{a_n\}$ is a \textbf{Cauchy sequence} if for every $\epsilon > 0$, there exists an $N \in \mathbb{N}$ such that for all $m, n \geq N$, 
\[
|a_m - a_n| < \epsilon.
\]

\textbf{Cauchy Criterion:} \\
A sequence of real numbers converges if and only if it is a Cauchy sequence.

This criterion is particularly useful when we don't know the limit of a sequence but want to prove it converges.

\textbf{Example:} \\
The sequence 
\[
a_n = \sum_{k=1}^{n} \frac{1}{k^2}
\]
is a Cauchy sequence because for $m > n$:
\[
|a_m - a_n| = \left|\sum_{k=n+1}^{m} \frac{1}{k^2}\right| < \sum_{k=n+1}^{m} \frac{1}{k^2} < \sum_{k=n+1}^{\infty} \frac{1}{k^2} < \frac{1}{n}.
\]
For large enough $n$, $\frac{1}{n} < \epsilon$, proving it's Cauchy.

\subsection{Subsequences and the Bolzano-Weierstrass Theorem} \label{subsequences}

A \textbf{subsequence} of $\{a_n\}$ is a sequence $\{a_{n_k}\}$ where $\{n_k\}$ is a strictly increasing sequence of natural numbers.

\textbf{Bolzano-Weierstrass Theorem:} \\
Every bounded sequence has a convergent subsequence.

This theorem is fundamental in analysis and is equivalent to the completeness of the real numbers.

\textbf{Examples of Subsequences:}
\begin{enumerate}[label=\textbf{\arabic*.}]
    \item For sequence $a_n = (-1)^n$, the subsequence $a_{2n} = 1$ converges to 1, and $a_{2n-1} = -1$ converges to -1.
    \item For sequence $a_n = \sin(n)$, while the sequence itself doesn't converge, we can extract convergent subsequences with various limits between -1 and 1.
\end{enumerate}

\textbf{Real-World Example - Population Dynamics:} \\
In population studies, a species' numbers might fluctuate seasonally, creating an oscillating sequence. By looking at subsequences (e.g., population every January), researchers can identify long-term trends.

\subsection{Series: Definition and Convergence} \label{series-definition}

A \textbf{series} is the formal sum of the terms of a sequence, denoted by
\[
\sum_{n=1}^{\infty} a_n.
\]
The \textbf{partial sums} of a series form a sequence $\{S_n\}$ where 
\[
S_n = \sum_{k=1}^{n} a_k.
\]
A series $\sum_{n=1}^{\infty} a_n$ \textbf{converges} if the sequence of partial sums $\{S_n\}$ converges to some limit $S$. We write
\[
\sum_{n=1}^{\infty} a_n = S.
\]
If the sequence of partial sums diverges, we say the series \textbf{diverges}.

\textbf{Examples:}
\begin{enumerate}[label=\textbf{\arabic*.}]
    \item \textbf{Geometric Series:} 
    \[
    \sum_{n=0}^{\infty} ar^n = \frac{a}{1-r} \quad \text{for } |r| < 1.
    \]
    \item \textbf{Harmonic Series:} 
    \[
    \sum_{n=1}^{\infty} \frac{1}{n} \quad (\text{diverges}).
    \]
    \item \textbf{p-Series:} 
    \[
    \sum_{n=1}^{\infty} \frac{1}{n^p} \quad (\text{converges for } p > 1,\ \text{diverges for } p \leq 1).
    \]
\end{enumerate}

\textbf{Real-World Example - Present Value of Perpetuity:} \\
In finance, a perpetuity is a stream of equal payments that continue indefinitely. If $C$ is the periodic payment and $r$ is the interest rate per period, the present value is:
\[
PV = \sum_{n=1}^{\infty} \frac{C}{(1+r)^n} = \frac{C}{r}.
\]
This is a convergent geometric series with first term $\frac{C}{1+r}$ and ratio $\frac{1}{1+r}$.

\subsection{Tests for Convergence} \label{convergence-tests}

\begin{enumerate}[label=\textbf{\arabic*.}]
    \item \textbf{Comparison Test}: \\
    If $0 \leq a_n \leq b_n$ for all $n \geq N$ and $\sum b_n$ converges, then $\sum a_n$ converges.
    
    \item \textbf{Limit Comparison Test}: \\
    If 
    \[
    \lim_{n \to \infty} \frac{a_n}{b_n} = L
    \]
    where $L > 0$ and $a_n, b_n > 0$, then $\sum a_n$ and $\sum b_n$ either both converge or both diverge.
    
    \item \textbf{Ratio Test}: \\
    If 
    \[
    \lim_{n \to \infty} \left|\frac{a_{n+1}}{a_n}\right| = r,
    \]
    then:
    \begin{itemize}
        \item If $r < 1$, the series converges absolutely.
        \item If $r > 1$ or $r = \infty$, the series diverges.
        \item If $r = 1$, the test is inconclusive.
    \end{itemize}
    
    \item \textbf{Root Test}: \\
    If 
    \[
    \lim_{n \to \infty} \sqrt[n]{|a_n|} = r,
    \]
    then:
    \begin{itemize}
        \item If $r < 1$, the series converges absolutely.
        \item If $r > 1$ or $r = \infty$, the series diverges.
        \item If $r = 1$, the test is inconclusive.
    \end{itemize}
    
    \item \textbf{Integral Test}: \\
    If $f$ is a positive, continuous, decreasing function with $f(n) = a_n$, then $\sum_{n=1}^{\infty} a_n$ converges if and only if 
    \[
    \int_{1}^{\infty} f(x) \, dx
    \]
    converges.
    
    \item \textbf{Alternating Series Test}: \\
    If $\{a_n\}$ is a sequence of positive terms such that:
    \begin{itemize}
        \item $a_{n+1} \leq a_n$ for all $n \geq N$,
        \item $\lim_{n \to \infty} a_n = 0$,
    \end{itemize}
    then the alternating series 
    \[
    \sum_{n=1}^{\infty} (-1)^{n+1} a_n
    \]
    converges.
\end{enumerate}

\textbf{Example Applications:}
\begin{enumerate}[label=\textbf{\arabic*.}]
    \item Testing $\sum_{n=1}^{\infty} \frac{1}{n^2 + 1}$ using the Comparison Test with $\frac{1}{n^2}$.
    \item Testing $\sum_{n=1}^{\infty} \frac{n!}{n^n}$ using the Ratio Test.
\end{enumerate}

\subsection{Absolute and Conditional Convergence} \label{absolute-convergence}

A series $\sum a_n$ \textbf{converges absolutely} if 
\[
\sum |a_n|
\]
converges.

A series $\sum a_n$ \textbf{converges conditionally} if $\sum a_n$ converges but $\sum |a_n|$ diverges.

\textbf{Properties:}
\begin{enumerate}[label=(\arabic*)]
    \item If a series converges absolutely, then it converges.
    \item The converse is not true; some series converge conditionally.
\end{enumerate}

\textbf{Example of Conditional Convergence:} \\
The alternating harmonic series 
\[
\sum_{n=1}^{\infty} \frac{(-1)^{n+1}}{n} = 1 - \frac{1}{2} + \frac{1}{3} - \frac{1}{4} + \ldots
\]
converges conditionally to $\ln(2)$. It converges by the Alternating Series Test, but 
\[
\sum_{n=1}^{\infty} \frac{1}{n}
\]
diverges.

\textbf{Real-World Example - Signal Processing:} \\
In signal processing, Fourier series decompose periodic signals into sums of sines and cosines. The convergence properties of these series determine how accurately the original signal can be reconstructed.

\subsection{Power Series and Radius of Convergence} \label{power-series}

A \textbf{power series} is a series of the form 
\[
\sum_{n=0}^{\infty} a_n (x-c)^n,
\]
where $c$ and $a_n$ are constants and $x$ is a variable.

The \textbf{radius of convergence} $R$ of a power series is the radius of the largest disk centered at $c$ in which the series converges for all $x$.

For a power series $\sum_{n=0}^{\infty} a_n (x-c)^n$:
\begin{itemize}
    \item If $|x-c| < R$, the series converges absolutely.
    \item If $|x-c| > R$, the series diverges.
    \item If $|x-c| = R$, the series may converge or diverge depending on the specific series.
\end{itemize}

\textbf{Methods to Find the Radius of Convergence:}
\begin{enumerate}[label=\textbf{\arabic*.}]
    \item \textbf{Ratio Test:} 
    \[
    R = \frac{1}{\lim_{n \to \infty} \left|\frac{a_{n+1}}{a_n}\right|} \quad (\text{if the limit exists}).
    \]
    \item \textbf{Root Test:} 
    \[
    R = \frac{1}{\lim_{n \to \infty} \sqrt[n]{|a_n|}} \quad (\text{if the limit exists}).
    \]
\end{enumerate}

\textbf{Examples:}
\begin{enumerate}[label=\textbf{\arabic*.}]
    \item For 
    \[
    \sum_{n=0}^{\infty} \frac{x^n}{n!},
    \]
    using the Ratio Test: 
    \[
    \lim_{n \to \infty} \left|\frac{a_{n+1}}{a_n}\right| = \lim_{n \to \infty} \frac{1}{n+1} = 0.
    \]
    So $R = \infty$, meaning this series (the exponential function $e^x$) converges for all $x$.
    
    \item For 
    \[
    \sum_{n=0}^{\infty} n! \, x^n,
    \]
    using the Ratio Test:
    \[
    \lim_{n \to \infty} \left|\frac{a_{n+1}}{a_n}\right| = \lim_{n \to \infty} (n+1) = \infty.
    \]
    So $R = 0$, meaning this series converges only at $x = 0$.
\end{enumerate}

\subsection{Applications of Sequences and Series} \label{sequences-applications}

\begin{enumerate}[label=\textbf{\arabic*.}]
    \item \textbf{Taylor Series:} Approximating functions using power series
    \[
    f(x) = \sum_{n=0}^{\infty} \frac{f^{(n)}(a)}{n!}(x-a)^n.
    \]
    
    \textit{Example:} 
    \[
    e^x = \sum_{n=0}^{\infty} \frac{x^n}{n!} = 1 + x + \frac{x^2}{2!} + \frac{x^3}{3!} + \ldots
    \]
    
    \item \textbf{Fourier Series:} Representing periodic functions as sums of sines and cosines
    \[
    f(x) = \frac{a_0}{2} + \sum_{n=1}^{\infty} \left[a_n \cos(nx) + b_n \sin(nx)\right].
    \]
    
    \item \textbf{Numerical Methods:}
    \begin{itemize}
        \item Euler's Method for solving differential equations.
        \item Newton's Method for finding roots.
        \item Approximating definite integrals using series.
    \end{itemize}
    
    \item \textbf{Physics:}
    \begin{itemize}
        \item Vibration analysis using Fourier series.
        \item Quantum mechanics wave functions.
        \item Statistical mechanics partition functions.
    \end{itemize}
    
    \item \textbf{Engineering:}
    \begin{itemize}
        \item Control systems using Z-transforms (discrete sequences).
        \item Digital signal processing.
        \item Circuit analysis using Laplace transforms.
    \end{itemize}
    
    \item \textbf{Finance:}
    \begin{itemize}
        \item Present value calculations.
        \item Bond pricing models.
        \item Annuity valuations.
    \end{itemize}
\end{enumerate}

\textbf{Real-World Example - Structural Engineering:} \\
Engineers use Fourier series to analyze the vibration modes of structures like bridges. By decomposing complex vibrations into simpler harmonic components, they can identify dangerous resonance frequencies and design structures to avoid them.

\bigskip

\section*{Conclusion}

Perfect sets, connected sets, and numerical sequences and series form the foundation of real analysis. These concepts not only provide mathematicians with powerful tools for theoretical exploration but also offer practical applications across numerous fields including physics, engineering, economics, and computer science.

The mathematical rigor we've developed in understanding these concepts allows us to model and solve complex real-world problems with precision. From designing electrical networks based on connected sets to calculating present values of financial instruments using convergent series, these mathematical tools have profound practical utility.

As you continue your studies in real analysis, remember that these concepts are interconnected. The properties of perfect sets inform our understanding of sequences in those sets; the convergence of series often depends on the connectedness of underlying domains; and the very completeness of $\mathbb{R}$ that makes the Bolzano-Weierstrass theorem possible is intimately related to perfect sets and their properties.

\end{document}
