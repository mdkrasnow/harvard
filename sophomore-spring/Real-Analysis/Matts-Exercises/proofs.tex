\documentclass[12pt,oneside]{article}
\usepackage[margin = 1in]{geometry}
\usepackage{graphicx}
\usepackage{color}
\usepackage{physics}
\usepackage{amsmath,amssymb,amsthm,mathtools}
\usepackage{fancyhdr}

\theoremstyle{definition}
\newtheorem{problem}{Problem}


\begin{document}
\pagestyle{fancy}
\fancyhead[L]{Math 112: Introductory Real Analysis}
\fancyhead[R]{Harvard University, Spring 2025}


\begin{center}
\bf \Large
Matt's Exercises \\[0.5 em]
\large
Due never!
\end{center}

\bigskip



\begin{problem}
(Exercise 1.1)

Imagine you have $n$ data points. Define d(x,y) to be the distance between x and y. 

d(x,y) = 1 if x and y are different, and d(x,y) = 0 if x and y are the same.

What is the smallest dimension $k$ such that the data points can be embedded in $\mathbb{R}^k$?


\end{problem}


\begin{problem}
(Exercise 1.2)

Can you create a mapping of the numbers between 0 and 1 to the real numbers such that the mapping is continuous?


\end{problem}   






\end{document}
