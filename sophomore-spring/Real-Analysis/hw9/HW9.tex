\documentclass[12pt,oneside]{article}
\usepackage[margin = 1in]{geometry}
\usepackage{graphicx}
\usepackage{color}
\usepackage{physics}
\usepackage{amsmath,amssymb,amsthm,mathtools}
\usepackage{fancyhdr}

\theoremstyle{definition}
\newtheorem{problem}{Problem}

\begin{document}
\pagestyle{fancy}
\fancyhead[L]{Math 112: Introductory Real Analysis}
\fancyhead[R]{Harvard University, Spring 2025}

\begin{center}
\bf \Large
Problem Set 9\\[0.5em]
\large
Due Wednesday, April 30, 2025
\end{center}

\bigskip

\begin{problem}[10 points]
Suppose $a \le x_0 \le b$ and
\[
f(x)=
\begin{cases}
1,&x=x_0,\\
0,&x\neq x_0.
\end{cases}
\]
Prove that $f$ is integrable on $[a,b]$ and that $\int_a^b f(x)\,dx=0$.
\end{problem}
\textbf{Proof.}
By Theorem 6.13 of Rudin, a bounded function with only finitely many points of discontinuity on a closed interval is Riemann integrable.  Here $f$ is discontinuous only at the single point $x_0$, so $f$ is integrable on $[a,b]$.

Next, given any $\varepsilon>0$, choose a partition $P$ of $[a,b]$ which includes $x_0$ as an endpoint of one subinterval of length $<\varepsilon$.  On that small subinterval $\Delta x<\varepsilon$, the supremum $M_i=1$ and on every other subinterval $M_j=0$.  Hence
\[
U(f,P)=1\cdot\Delta x+0\le\varepsilon,
\qquad
L(f,P)=0.
\]
Since $\overline{\int_a^b}f\le U(f,P)<\varepsilon$ and $\underline{\int_a^b}f=0$, letting $\varepsilon\to0$ shows
\[
\int_a^b f(x)\,dx=0.
\]
\hfill$\square$

\bigskip

\begin{problem}[10 points]
Suppose $f$ is continuous on $[a,b]$, $f\ge0$, and $\displaystyle\int_a^b f(x)\,dx=0$. Prove that $f(x)=0$ for all $x\in[a,b]$.
\end{problem}
\textbf{Proof.}
By the Extreme Value Theorem (Rudin Thm 4.18), $f$ attains a minimum and maximum on $[a,b]$.  If there were some $c\in[a,b]$ with $f(c)=\delta>0$, then by continuity there exists $\eta>0$ such that $f(x)>\delta/2$ for all $x\in(c-\eta,c+\eta)\cap[a,b]$.  Hence
\[
\int_a^b f(x)\,dx
\ge \int_{c-\eta}^{c+\eta} f(x)\,dx
\ge \int_{c-\eta}^{c+\eta}\!\frac\delta2\,dx
= \delta\eta >0,
\]
contradicting the hypothesis.  Therefore no such $c$ exists and $f\equiv0$.
\hfill$\square$

\bigskip

\begin{problem}[10 points]
Let
\[
f(x)=
\begin{cases}
1,&x\in\mathbb{Q},\\
0,&x\notin\mathbb{Q}.
\end{cases}
\]
Prove that $f$ is not integrable on $[a,b]$ for any $a<b$.
\end{problem}
\textbf{Proof.}
On every nonempty subinterval $[x_{i-1},x_i]\subset[a,b]$, both rationals and irrationals are dense.  Thus for each $i$, the supremum $M_i=1$ and the infimum $m_i=0$.  For any partition $P$, 
\[
U(f,P)=\sum_i 1\cdot\Delta x_i = b-a,
\qquad
L(f,P)=\sum_i 0\cdot\Delta x_i =0.
\]
Hence the upper integral is $b-a$ and the lower integral is $0$, so they do not agree.  Therefore $f$ is not Riemann integrable.
\hfill$\square$

\bigskip

\begin{problem}[10 points]
Suppose $f$ is bounded on $[a,b]$ and $f^2$ is integrable on $[a,b]$.
\begin{enumerate}
  \item Does it follow that $f$ is integrable on $[a,b]$?
  \item Does the answer change if instead we assume $f^3$ is integrable?
\end{enumerate}
Justify your answers.
\end{problem}
\textbf{Answer.}
\begin{enumerate}
\item Not in general.  Consider the function
\[
g(x)=
\begin{cases}
1,&x\in\mathbb{Q},\\
-1,&x\notin\mathbb{Q},
\end{cases}
\]
on $[a,b]$.  Then $g^2(x)\equiv1$, so $g^2$ is integrable, but $g$ is discontinuous everywhere, hence not integrable.

\item Yes, if $f^3$ is integrable then $f$ must be integrable.  The mapping $\varphi(u)=u^3$ has derivative $\varphi'(u)=3u^2$, which is bounded on the range of $f$ since $f$ is bounded.  Thus $\varphi$ is Lipschitz on that range, and by the Lipschitz composition theorem (Rudin Thm 6.15), $f^3=\varphi\circ f$ integrable implies $f$ integrable.
\end{enumerate}
\hfill$\square$

\bigskip

\begin{problem}[Extra Credit; 10 points]
For any integrable $f$ on $[a,b]$ define
\[
\|f\|_2 := \Bigl(\int_a^b |f(x)|^2\,dx\Bigr)^{1/2}.
\]
For integrable $f,g,h$, prove the triangle inequality
\[
\|f-h\|_2 \;\le\;\|f-g\|_2 + \|g-h\|_2.
\]
\end{problem}
\textbf{Proof.}
Let
\[
u = f-g,\quad v = g-h,
\]
so $f-h = u+v$.  Then
\[
\|u+v\|_2^2
= \int_a^b (u+v)^2
= \int_a^b u^2 + 2uv + v^2
= \|u\|_2^2 + 2\langle u,v\rangle + \|v\|_2^2.
\]
By the Cauchy–Schwarz inequality (Rudin Thm 3.14),
\[
|\langle u,v\rangle|
\le \|u\|_2\,\|v\|_2,
\]
so
\[
\|u+v\|_2^2
\le \|u\|_2^2 + 2\|u\|_2\|v\|_2 + \|v\|_2^2
= \bigl(\|u\|_2 + \|v\|_2\bigr)^2.
\]
Taking square roots yields
\[
\|f-h\|_2 = \|u+v\|_2 \le \|u\|_2 + \|v\|_2
= \|f-g\|_2 + \|g-h\|_2.
\]
\hfill$\square$

\end{document}
