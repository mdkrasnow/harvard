\documentclass[12pt,oneside]{article}
\usepackage[margin = 1in]{geometry}
\usepackage{graphicx}
\usepackage{color}
\usepackage{physics}
\usepackage{amsmath,amssymb,amsthm,mathtools}
\usepackage{fancyhdr}

\theoremstyle{definition}
\newtheorem{problem}{Problem}
\newtheorem*{solution}{Solution} % Using unnumbered solution environment
\renewcommand{\qedsymbol}{$\blacksquare$} % Change proof end symbol

% Reference to Rudin
\newcommand{\rudin}{\textit{Principles of Mathematical Analysis} by Walter Rudin (3rd Ed.)}

\begin{document}
\pagestyle{fancy}
\fancyhead[L]{Math 112: Introductory Real Analysis}
\fancyhead[R]{Harvard University, Spring 2025}


\begin{center}
\bf \Large
Problem Set 5 \\[0.5 em]
\large
Due Wednesday, April 2, 2025
\end{center}

\bigskip

\begin{problem}[10 points]
Recall the definition of the upper limit $\limsup_{n\rightarrow \infty}a_n$ from Definition 3.16 (or its equivalent formulation in Theorem 3.17) of Rudin. 
This exercise explains why the upper limit is called ``$\limsup$''.
Suppose that $\{a_n\}$ is a real sequence such that $\limsup_{n\rightarrow \infty}a_n \in \mathbb{R}$. 
Prove that
\[
\lim_{n\rightarrow \infty} \sup\{a_j\;\vert\; j\geq n\} = 
\limsup_{n\rightarrow \infty}a_n. 
\]
\end{problem}

\begin{solution}
Let $\{a_n\}$ be a real sequence and let $a^* = \limsup_{n\rightarrow \infty}a_n$. We are given that $a^* \in \mathbb{R}$.
Let $s_n = \sup\{a_j \mid j \ge n\}$.
The sequence $\{s_n\}$ is non-increasing. Indeed, the set $\{a_j \mid j \ge n+1\}$ is a subset of $\{a_j \mid j \ge n\}$, so the supremum over the smaller set cannot exceed the supremum over the larger set, i.e., $s_{n+1} \le s_n$.
Since $\{s_n\}$ is a non-increasing sequence of real numbers, it converges to a limit $L \in \mathbb{R} \cup \{-\infty\}$. We want to prove that $L = a^*$.

We will use Theorem 3.17 from \rudin{}, which states that $a^*$ is the unique real number such that:
\begin{itemize}
    \item[(a)] For every $\epsilon > 0$, there exists an integer $N$ such that $n \geq N$ implies $a_n < a^* + \epsilon$.
    \item[(b)] For every $\epsilon > 0$, and for every integer $N$, there exists an integer $n \geq N$ such that $a_n > a^* - \epsilon$.
\end{itemize}
We will show that $L = \lim_{n\to\infty} s_n$ satisfies these two properties.

\textbf{Proof that $L$ satisfies property (a):}
Since $s_k = \sup\{a_j \mid j \ge k\}$, we have $a_k \le s_k$ for all $k$.
Since $s_n \to L$ and $\{s_n\}$ is non-increasing, we have $s_n \ge L$ for all $n$.
Also, because $s_n \to L$, for any $\epsilon > 0$, there exists an integer $N$ such that for all $n \ge N$, $s_n < L + \epsilon$.
Combining these, for $n \ge N$, we have $a_n \le s_n < L + \epsilon$.
Thus, $L$ satisfies property (a) (with $L$ in place of $a^*$).

\textbf{Proof that $L$ satisfies property (b):}
Let $\epsilon > 0$ and let $N$ be any integer. We need to show there exists $n \ge N$ such that $a_n > L - \epsilon$.
Since $L = \lim_{k\to\infty} s_k = \inf_{k\ge 1} s_k$, we know $s_N = \sup\{a_j \mid j \ge N\} \ge L$.
By the definition of the supremum, for the set $\{a_j \mid j \ge N\}$ and the number $s_N - \epsilon/2 < s_N$, there must exist an element $a_n$ in the set (so $n \ge N$) such that $a_n > s_N - \epsilon/2$.
Since $s_N \ge L$, we have $a_n > s_N - \epsilon/2 \ge L - \epsilon/2$.
Since $L - \epsilon/2 > L - \epsilon$, we have found an $n \ge N$ such that $a_n > L - \epsilon$.
Thus, $L$ satisfies property (b).

\textbf{Conclusion:}
Since $L = \lim_{n\to\infty} s_n$ satisfies both properties (a) and (b) from Theorem 3.17 of Rudin, and $a^*$ is the unique number with these properties, it must be that $L = a^*$.
Therefore,
\[
\lim_{n\rightarrow \infty} \sup\{a_j\;\vert\; j\geq n\} = 
\limsup_{n\rightarrow \infty}a_n. 
\]
This justifies calling the upper limit "lim sup".
\end{solution}
\qed


\bigskip
\bigskip

\begin{problem}[10 points]
For any two real sequences $\{a_n\}$ and $\{b_n\}$, prove that
\[
\limsup_{n\rightarrow \infty}(a_n + b_n) \leq \limsup_{n\rightarrow \infty} a_n + \limsup_{n\rightarrow \infty} b_n,
\]
provided the sum on the right is not of the form $\infty - \infty$. 
\end{problem}

\begin{solution}
Let $a^* = \limsup_{n\rightarrow \infty} a_n$ and $b^* = \limsup_{n\rightarrow \infty} b_n$. Let $c^* = \limsup_{n\rightarrow \infty}(a_n + b_n)$. We want to show $c^* \le a^* + b^*$, assuming $a^*+b^*$ is well-defined (not $\infty-\infty$).

We use the characterization established in Problem 1 (which is also a standard definition of lim sup):
$a^* = \lim_{n\to\infty} \sup\{a_k \mid k \ge n\}$ and $b^* = \lim_{n\to\infty} \sup\{b_k \mid k \ge n\}$.
Let $s_n(x) = \sup\{x_k \mid k \ge n\}$. Then $a^* = \lim_{n\to\infty} s_n(a)$, $b^* = \lim_{n\to\infty} s_n(b)$, and $c^* = \lim_{n\to\infty} s_n(a+b)$.

Consider the term $s_n(a+b) = \sup\{a_k + b_k \mid k \ge n\}$.
For any $k \ge n$, we have $a_k \le s_n(a)$ and $b_k \le s_n(b)$.
Therefore, $a_k + b_k \le s_n(a) + s_n(b)$ for all $k \ge n$.
This means that $s_n(a) + s_n(b)$ is an upper bound for the set $\{a_k + b_k \mid k \ge n\}$.
By the definition of the supremum, $s_n(a+b)$ must be less than or equal to any upper bound. Thus,
\[ s_n(a+b) \le s_n(a) + s_n(b). \]
This inequality holds for all $n$. Now we take the limit as $n \to \infty$.
The sequences $s_n(a)$, $s_n(b)$, and $s_n(a+b)$ are non-increasing and hence their limits exist in the extended real number system $\mathbb{R} \cup \{+\infty, -\infty\}$.
Let $L = \lim_{n\to\infty} (s_n(a) + s_n(b))$. By standard limit theorems (e.g., Theorem 3.20(b) in Rudin, adapted for extended real numbers), if the sum $\lim s_n(a) + \lim s_n(b) = a^* + b^*$ is defined (not $\infty-\infty$), then $L = a^* + b^*$.
Using the inequality $s_n(a+b) \le s_n(a) + s_n(b)$ and taking the limit (Theorem 3.20(a) in Rudin, adapted), we get:
\[ \lim_{n\to\infty} s_n(a+b) \le \lim_{n\to\infty} (s_n(a) + s_n(b)). \]
Substituting the definitions of $c^*$ and the limits of $s_n(a)$ and $s_n(b)$:
\[ c^* \le a^* + b^*. \]
This holds provided $a^*+b^*$ is not of the form $\infty - \infty$.

Let's briefly check the infinite cases covered by this argument:
\begin{itemize}
    \item If $a^* = \infty$, then $a^*+b^*$ is $\infty$ (since $b^* \ne -\infty$). The inequality $c^* \le \infty$ is always true.
    \item If $b^* = \infty$, then $a^*+b^*$ is $\infty$ (since $a^* \ne -\infty$). The inequality $c^* \le \infty$ is always true.
    \item If $a^* = -\infty$ and $b^* = -\infty$, then $a^*+b^* = -\infty$. The inequality becomes $c^* \le -\infty$, which implies $c^* = -\infty$. This is correct, as if $a_n \to -\infty$ and $b_n \to -\infty$, then $a_n+b_n \to -\infty$, so $c^* = -\infty$.
    \item If $a^* = -\infty$ and $b^*$ is finite, then $a^*+b^* = -\infty$. The inequality becomes $c^* \le -\infty$, implying $c^*=-\infty$. This is correct, as if $a_n \to -\infty$ and $b_n$ is bounded above for large $n$, then $a_n+b_n \to -\infty$, so $c^* = -\infty$. (Symmetrically if $a^*$ finite and $b^*=-\infty$).
\end{itemize}
The proof holds in all cases where the sum $a^*+b^*$ is defined.
\end{solution}
\qed


\bigskip
\bigskip

\begin{problem}[10 points]
Use the Root Test or the Ratio Test to determine which of the following series converge. 
\begin{itemize}
\item[(a)] $\displaystyle \sum_{n=1}^{\infty} \frac{1}{n!}$
\item[(b)] $\displaystyle \sum_{n=1}^{\infty} \frac{(-1)^n}{n^n}$
\item[(c)] $\displaystyle \sum_{n=1}^{\infty} \frac{n^{100}}{n!}$
\item[(d)] $\displaystyle \sum_{n=1}^{\infty} \frac{(-1)^n n^n}{n!}$
\end{itemize}
\end{problem}

\begin{solution}
We will use the Ratio Test (Theorem 3.34, \rudin{}) and the Root Test (Theorem 3.33, \rudin{}). Recall that for a series $\sum a_n$:
\begin{itemize}
    \item Ratio Test: Examines $L = \lim_{n\to\infty} |a_{n+1}/a_n|$. Converges if $L < 1$, diverges if $L > 1$. Inconclusive if $L=1$. More generally, converges if $\limsup |a_{n+1}/a_n| < 1$, diverges if $\liminf |a_{n+1}/a_n| > 1$.
    \item Root Test: Examines $\alpha = \limsup_{n\to\infty} \sqrt[n]{|a_n|}$. Converges if $\alpha < 1$, diverges if $\alpha > 1$. Inconclusive if $\alpha=1$.
\end{itemize}
Absolute convergence implies convergence (Theorem 3.45, Rudin).

\medskip

\textbf{(a) $\displaystyle \sum_{n=1}^{\infty} \frac{1}{n!}$}
Let $a_n = 1/n!$. The terms are positive. We apply the Ratio Test:
\[
\lim_{n\to\infty} \left| \frac{a_{n+1}}{a_n} \right| = \lim_{n\to\infty} \frac{1/(n+1)!}{1/n!} = \lim_{n\to\infty} \frac{n!}{(n+1)!} = \lim_{n\to\infty} \frac{n!}{(n+1)n!} = \lim_{n\to\infty} \frac{1}{n+1} = 0.
\]
Since the limit $L=0 < 1$, the series converges by the Ratio Test.

\medskip

\textbf{(b) $\displaystyle \sum_{n=1}^{\infty} \frac{(-1)^n}{n^n}$}
Let $a_n = (-1)^n / n^n$. We apply the Root Test to $|a_n| = 1/n^n$:
\[
\limsup_{n\to\infty} \sqrt[n]{|a_n|} = \limsup_{n\to\infty} \sqrt[n]{\frac{1}{n^n}} = \limsup_{n\to\infty} \left( \left(\frac{1}{n}\right)^n \right)^{1/n} = \limsup_{n\to\infty} \frac{1}{n} = 0.
\]
Since the limit $\alpha=0 < 1$, the series $\sum a_n$ converges absolutely by the Root Test, and therefore converges.

\medskip

\textbf{(c) $\displaystyle \sum_{n=1}^{\infty} \frac{n^{100}}{n!}$}
Let $a_n = n^{100}/n!$. The terms are positive. We apply the Ratio Test:
\begin{align*}
\lim_{n\to\infty} \left| \frac{a_{n+1}}{a_n} \right| &= \lim_{n\to\infty} \frac{(n+1)^{100}/(n+1)!}{n^{100}/n!} \\
&= \lim_{n\to\infty} \frac{(n+1)^{100}}{(n+1)!} \cdot \frac{n!}{n^{100}} \\
&= \lim_{n\to\infty} \frac{(n+1)^{100}}{(n+1)n!} \cdot \frac{n!}{n^{100}} \\
&= \lim_{n\to\infty} \frac{(n+1)^{100}}{(n+1)n^{100}} \\
&= \lim_{n\to\infty} \frac{1}{n+1} \left( \frac{n+1}{n} \right)^{100} \\
&= \lim_{n\to\infty} \frac{1}{n+1} \left( 1 + \frac{1}{n} \right)^{100}.
\end{align*}
As $n\to\infty$, $\frac{1}{n+1} \to 0$ and $(1 + 1/n)^{100} \to (1+0)^{100} = 1$.
So the limit is $L = 0 \times 1 = 0$.
Since $L=0 < 1$, the series converges by the Ratio Test.

\medskip

\textbf{(d) $\displaystyle \sum_{n=1}^{\infty} \frac{(-1)^n n^n}{n!}$}
Let $a_n = (-1)^n n^n / n!$. We consider $|a_n| = n^n / n!$. Let's try the Ratio Test on $|a_n|$:
\begin{align*}
\lim_{n\to\infty} \left| \frac{a_{n+1}}{a_n} \right| &= \lim_{n\to\infty} \frac{|a_{n+1}|}{|a_n|} = \lim_{n\to\infty} \frac{(n+1)^{n+1}/(n+1)!}{n^n/n!} \\
&= \lim_{n\to\infty} \frac{(n+1)^{n+1}}{(n+1)!} \cdot \frac{n!}{n^n} \\
&= \lim_{n\to\infty} \frac{(n+1)^{n+1}}{(n+1)n!} \cdot \frac{n!}{n^n} \\
&= \lim_{n\to\infty} \frac{(n+1)^n}{n^n} = \lim_{n\to\infty} \left( \frac{n+1}{n} \right)^n \\
&= \lim_{n\to\infty} \left( 1 + \frac{1}{n} \right)^n = e.
\end{align*}
Since the limit $L=e \approx 2.718 > 1$, the series $\sum |a_n|$ diverges by the Ratio Test.
The Ratio Test result $L>1$ implies that $|a_n|$ does not tend to 0. In fact, since $\lim |a_{n+1}|/|a_n| = e > 1$, $|a_n| \to \infty$.
Since the terms $a_n = (-1)^n |a_n|$ do not converge to 0 (they oscillate between large positive and negative values), the series $\sum a_n$ diverges by the Term Test (Theorem 3.23, Rudin).

(a) Converges. (b) Converges. (c) Converges. (d) Diverges.
\end{solution}
\qed


\bigskip
\bigskip

\begin{problem}[10 points]
Let $\{a_n\}$ be a sequence of non-negative real numbers. 
Prove that the convergence of $\sum_{n=1}^{\infty} a_n$ implies the convergence of 
\[
\sum_{n=1}^{\infty} \frac{\sqrt{a_n}}{n}.
\]
\end{problem}

\begin{solution}
We are given that $a_n \ge 0$ for all $n$ and that the series $\sum_{n=1}^{\infty} a_n$ converges.
We want to show that the series $\sum_{n=1}^{\infty} \frac{\sqrt{a_n}}{n}$ converges.

We use the Arithmetic Mean - Geometric Mean (AM-GM) inequality, which states that for non-negative real numbers $x$ and $y$, $\sqrt{xy} \le \frac{x+y}{2}$.
Let $x = a_n$ and $y = \frac{1}{n^2}$. Both are non-negative.
Applying the AM-GM inequality:
\[
\sqrt{a_n \cdot \frac{1}{n^2}} \le \frac{a_n + \frac{1}{n^2}}{2}
\]
\[
\frac{\sqrt{a_n}}{\sqrt{n^2}} \le \frac{a_n}{2} + \frac{1}{2n^2}
\]
Since $n \ge 1$, $\sqrt{n^2} = n$. So,
\[
\frac{\sqrt{a_n}}{n} \le \frac{1}{2} a_n + \frac{1}{2n^2}.
\]
We know that $\sum_{n=1}^{\infty} a_n$ converges by hypothesis.
We also know that the series $\sum_{n=1}^{\infty} \frac{1}{n^2}$ converges. This is a p-series with $p=2 > 1$ (see Rudin, Theorem 3.28).
Since $\sum a_n$ and $\sum 1/n^2$ converge, their scalar multiples also converge, and their sum converges (Rudin, Theorem 3.47):
\[
\sum_{n=1}^{\infty} \left( \frac{1}{2} a_n + \frac{1}{2n^2} \right) = \frac{1}{2} \sum_{n=1}^{\infty} a_n + \frac{1}{2} \sum_{n=1}^{\infty} \frac{1}{n^2}.
\]
Thus, the series $\sum (\frac{1}{2} a_n + \frac{1}{2n^2})$ converges.

Let $b_n = \frac{\sqrt{a_n}}{n}$ and $c_n = \frac{1}{2} a_n + \frac{1}{2n^2}$. We have shown that $0 \le b_n \le c_n$ for all $n \ge 1$, and that $\sum_{n=1}^{\infty} c_n$ converges.
By the Comparison Test (Rudin, Theorem 3.25(a)), since the terms $b_n$ are non-negative and are bounded above by the terms of a convergent series $\sum c_n$, the series $\sum_{n=1}^{\infty} b_n = \sum_{n=1}^{\infty} \frac{\sqrt{a_n}}{n}$ must also converge.
\end{solution}
\qed


\bigskip
\bigskip

\begin{problem}[Extra Credit; 10 points]
Let $\{a_n\}$ be a sequence of positive numbers and let $\{b_n\}$ be a convergent sequence of positive numbers with nonzero limit. 
Prove that
\[
\limsup_{n\rightarrow \infty} a_nb_n = \limsup_{n\rightarrow \infty}a_n \lim_{n\rightarrow \infty}b_n.
\]
\end{problem}

\begin{solution}
Let $a^* = \limsup_{n\rightarrow \infty} a_n$. Since $a_n > 0$, we have $a^* \ge 0$. Note that $a^*$ could be $+\infty$.
Let $L = \lim_{n\rightarrow \infty} b_n$. We are given that $b_n > 0$ for all $n$, and $L \in \mathbb{R}$ with $L > 0$.
Let $c^* = \limsup_{n\rightarrow \infty} (a_n b_n)$. We want to prove $c^* = a^* L$.

We will prove the equality by showing $c^* \le a^* L$ and $c^* \ge a^* L$.

\textbf{Case 1: $a^*$ is finite ($0 \le a^* < \infty$).}

\textit{Proof of $c^* \le a^* L$:}
Let $\epsilon > 0$. Since $L > 0$ and $a^* \ge 0$, we can choose $\delta > 0$ small enough for our purposes later.
Since $a^* = \limsup a_n$, by Theorem 3.17(a) of Rudin, there exists $N_a$ such that for $n \ge N_a$, $a_n < a^* + \delta$.
Since $b_n \to L$, there exists $N_b$ such that for $n \ge N_b$, $|b_n - L| < \delta$, which implies $L - \delta < b_n < L + \delta$. Since $L>0$, we can choose $\delta$ small enough such that $L-\delta > 0$. So $0 < b_n < L + \delta$.
Let $N = \max(N_a, N_b)$. For $n \ge N$, we have $a_n > 0$ and $b_n > 0$, so $a_n b_n > 0$. Also,
\[ a_n b_n < (a^* + \delta)(L + \delta) = a^* L + \delta(a^* + L) + \delta^2. \]
Let $s_n = \sup\{a_k b_k \mid k \ge n\}$. For $n \ge N$, we have $s_n \le a^* L + \delta(a^* + L) + \delta^2$.
Taking the limit as $n \to \infty$, using the result from Problem 1 ($c^* = \lim s_n$), we get
\[ c^* = \lim_{n\to\infty} s_n \le a^* L + \delta(a^* + L) + \delta^2. \]
This inequality holds for any sufficiently small $\delta > 0$. We can make the term $\delta(a^* + L) + \delta^2$ arbitrarily small by choosing $\delta$ small. Specifically, for any $\epsilon > 0$, we can choose $\delta$ such that $\delta(a^* + L) + \delta^2 < \epsilon$.
This implies $c^* \le a^* L + \epsilon$ for any $\epsilon > 0$. Therefore, we must have $c^* \le a^* L$.

\textit{Proof of $c^* \ge a^* L$:}
Since $a^* = \limsup a_n$, there exists a subsequence $\{a_{n_k}\}$ such that $a_{n_k} \to a^*$ as $k \to \infty$ (Definition 3.16 and Theorem 3.7 of Rudin).
Since $b_n \to L$, any subsequence of $\{b_n\}$ must also converge to $L$. In particular, $b_{n_k} \to L$ as $k \to \infty$.
Consider the subsequence $\{a_{n_k} b_{n_k}\}$ of $\{a_n b_n\}$. By the limit properties for sequences (Theorem 3.20(d) of Rudin),
\[ \lim_{k\to\infty} (a_{n_k} b_{n_k}) = (\lim_{k\to\infty} a_{n_k}) (\lim_{k\to\infty} b_{n_k}) = a^* L. \]
We have found a subsequential limit of $\{a_n b_n\}$ that equals $a^* L$.
The $\limsup$ of a sequence is the supremum of its subsequential limits (Definition 3.16, Rudin). Therefore,
\[ c^* = \limsup_{n\rightarrow \infty} (a_n b_n) \ge a^* L. \]

Combining $c^* \le a^* L$ and $c^* \ge a^* L$, we conclude that $c^* = a^* L$ when $a^*$ is finite.

\textbf{Case 2: $a^* = +\infty$.}
We need to show $c^* = \infty \cdot L = \infty$.
Since $a^* = \limsup a_n = \infty$, there exists a subsequence $\{a_{n_k}\}$ such that $a_{n_k} \to \infty$ as $k \to \infty$.
As before, since $b_n \to L > 0$, the subsequence $b_{n_k} \to L$.
Consider the subsequence $\{a_{n_k} b_{n_k}\}$. We want to show $a_{n_k} b_{n_k} \to \infty$.
Since $b_{n_k} \to L > 0$, for $\epsilon = L/2 > 0$, there exists $K_1$ such that for $k \ge K_1$, $|b_{n_k} - L| < L/2$, which implies $b_{n_k} > L - L/2 = L/2$.
Since $a_{n_k} \to \infty$, for any $M > 0$, there exists $K_2$ such that for $k \ge K_2$, $a_{n_k} > \frac{2M}{L}$.
Let $K = \max(K_1, K_2)$. For $k \ge K$, we have
\[ a_{n_k} b_{n_k} > \left( \frac{2M}{L} \right) \cdot \left( \frac{L}{2} \right) = M. \]
Since for any $M > 0$, we can find $K$ such that $a_{n_k} b_{n_k} > M$ for all $k \ge K$, this means $\lim_{k\to\infty} a_{n_k} b_{n_k} = +\infty$.
We have found a subsequence of $\{a_n b_n\}$ that diverges to $+\infty$. This means $+\infty$ is a subsequential limit of $\{a_n b_n\}$.
The $\limsup$ is the supremum of all subsequential limits. Therefore,
\[ c^* = \limsup_{n\rightarrow \infty} (a_n b_n) \ge +\infty. \]
Since the $\limsup$ cannot exceed $+\infty$, we must have $c^* = +\infty$.
This matches the expected result $a^* L = \infty \cdot L = \infty$ (since $L>0$).

In both cases ($a^*$ finite and $a^* = \infty$), we have shown that $\limsup (a_n b_n) = (\limsup a_n) (\lim b_n)$.
\end{solution}
\qed

\end{document}