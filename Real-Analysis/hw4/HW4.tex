\documentclass[12pt,oneside]{article}
\usepackage[margin=1in]{geometry}
\usepackage{graphicx}
\usepackage{color}
\usepackage{physics}
\usepackage{amsmath,amssymb,amsthm,mathtools}
\usepackage{fancyhdr}
\usepackage[most]{tcolorbox}

\theoremstyle{definition}
\newtheorem{problem}{Problem}

% Define a new tcolorbox environment for solution spaces
\newtcolorbox{solution}{
  breakable,
  enhanced,
  colback=teal!10,      % light teal background
  colframe=teal!70!black, % darker teal frame
  title=Solution
}

\pagestyle{fancy}
\fancyhead[L]{Math 112: Introductory Real Analysis}
\fancyhead[R]{Harvard University, Spring 2025}

\begin{document}
\begin{center}
  \bf \Large
  Problem Set 4 \\[0.5em]
  \large
  Due Wednesday, March 5, 2025
\end{center}

\bigskip

\begin{problem}
(Exercise 2.12 in Rudin)\\
Let \(K \subset \mathbb{R}\) consist of \(0\) and the numbers \(\frac{1}{n}\) for \(n = 1, 2, 3, \ldots\); that is,
\[
K = \{0\} \cup \left\{\frac{1}{n} : n \in \mathbb{N}\right\}.
\]
Prove directly from the definition (without using the Heine--Borel theorem) that \(K\) is compact.
\end{problem}

\newpage
\begin{solution}
Let \(\mathcal{U}\) be an arbitrary open cover of \(K\). Since \(0 \in K\), there exists some open set \(U_0 \in \mathcal{U}\) with \(0 \in U_0\). Because \(U_0\) is open, there exists \(\varepsilon > 0\) such that
\[
(-\varepsilon, \varepsilon) \subset U_0.
\]
Since \(\lim_{n\to\infty} \frac{1}{n} = 0\), there exists \(N \in \mathbb{N}\) such that for all \(n \ge N\),
\[
\frac{1}{n} < \varepsilon.
\]
Thus, for all \(n \ge N\), we have \(\frac{1}{n} \in (-\varepsilon, \varepsilon) \subset U_0\).

The finitely many points \(\frac{1}{1}, \frac{1}{2}, \ldots, \frac{1}{N-1}\) are each contained in some members \(U_1, U_2, \ldots, U_{N-1}\) of \(\mathcal{U}\). Therefore, the finite collection
\[
\{U_0, U_1, U_2, \ldots, U_{N-1}\}
\]
covers \(K\). Since every open cover of \(K\) has a finite subcover, \(K\) is compact.
\end{solution}

\bigskip

\newpage
\begin{problem}
(Exercise 2.14 in Rudin)\\
Give an example of an open cover of the open interval \((0,1) \subset \mathbb{R}\) which has no finite subcover (and prove that property).
\end{problem}

\newpage
\begin{solution}
Consider the collection
\[
\mathcal{U} = \left\{ \left(\frac{1}{n}, 1\right) : n \in \mathbb{N},\, n\geq 2 \right\}.
\]
For any \(x \in (0,1)\), choose \(n \in \mathbb{N}\) such that \(n > \frac{1}{x}\). Then
\[
\frac{1}{n} < x,
\]
so \(x \in \left(\frac{1}{n}, 1\right)\). Hence, \(\mathcal{U}\) is an open cover of \((0,1)\).

Now, assume for contradiction that there exists a finite subcover
\[
\left\{\left(\frac{1}{n_1},1\right), \left(\frac{1}{n_2},1\right), \ldots, \left(\frac{1}{n_k},1\right)\right\}.
\]
Let \(N = \max\{n_1,n_2,\ldots,n_k\}\). Then every interval in this finite collection is of the form \(\left(\frac{1}{n_i},1\right)\) with \(\frac{1}{n_i} \ge \frac{1}{N}\). Therefore, none of these intervals can cover any point less than or equal to \(\frac{1}{N}\). In particular, the point
\[
x = \frac{1}{N+1}
\]
satisfies \(x < \frac{1}{N}\) and hence is not contained in any \(\left(\frac{1}{n_i},1\right)\), a contradiction. Thus, no finite subcover exists.
\end{solution}

\bigskip

\newpage
\begin{problem}
(Exercise 2.15 in Rudin)
\begin{enumerate}
  \item Construct a collection of closed subsets \(\{K_{\alpha}\}_{\alpha \in J}\) of \(\mathbb{R}\) such that the intersection of every (nonempty) finite subcollection is nonempty, while 
  \[
  \bigcap_{\alpha \in J} K_\alpha = \emptyset.
  \]
  \item Do the same with the word ``closed'' replaced by ``bounded''.
\end{enumerate}
\end{problem}

\newpage
\begin{solution}
\textbf{(a) Closed subsets:} \\
For each \(n \in \mathbb{N}\), define
\[
K_n = [n, \infty).
\]
Each \(K_n\) is closed in \(\mathbb{R}\). For any finite subcollection \(\{K_{n_1}, K_{n_2}, \ldots, K_{n_k}\}\), let
\[
N = \max\{n_1, n_2, \ldots, n_k\}.
\]
Then,
\[
\bigcap_{i=1}^k K_{n_i} = [N, \infty),
\]
which is nonempty. However,
\[
\bigcap_{n=1}^\infty K_n = \bigcap_{n=1}^\infty [n,\infty) = \emptyset,
\]
since no real number is greater than or equal to every natural number.

\bigskip

\textbf{(b) Bounded subsets:} \\
For each \(n \in \mathbb{N}\), define
\[
B_n = \left(0, \frac{1}{n}\right).
\]
Each \(B_n\) is bounded. For any finite subcollection \(\{B_{n_1}, B_{n_2}, \ldots, B_{n_k}\}\), let
\[
N = \max\{n_1, n_2, \ldots, n_k\}.
\]
Then,
\[
\bigcap_{i=1}^k B_{n_i} = \left(0, \frac{1}{N}\right),
\]
which is nonempty. However, the intersection over all \(n\) is
\[
\bigcap_{n=1}^\infty B_n = \{x \in \mathbb{R} : 0 < x < \frac{1}{n} \text{ for all } n\in\mathbb{N}\} = \emptyset,
\]
since for any \(x>0\) there exists \(n\) such that \(\frac{1}{n} < x\).
\end{solution}

\bigskip

\newpage
\begin{problem}
Let \(A\) and \(B\) be connected subsets of a metric space \(X\). If \(A \cap B \neq \emptyset\), prove that \(A \cup B\) is connected.
\end{problem}

\newpage
\begin{solution}
Assume, to the contrary, that \(A \cup B\) is disconnected. Then there exist nonempty disjoint open sets \(U\) and \(V\) in \(X\) such that
\[
A \cup B = U \cup V.
\]
Since \(A\) is connected, it must lie entirely in one of these open sets; without loss of generality, assume
\[
A \subset U.
\]
Similarly, since \(B\) is connected, we must have either \(B \subset U\) or \(B \subset V\). However, because \(A \cap B \neq \emptyset\), there exists a point \(x \in A \cap B\). Since \(x \in A\) and \(A \subset U\), it follows that \(x \in U\). Thus, \(B\) cannot be entirely contained in \(V\); hence, \(B \subset U\) as well. This implies
\[
A \cup B \subset U,
\]
which contradicts the fact that \(V\) is nonempty. Therefore, \(A \cup B\) must be connected.
\end{solution}

\end{document}
