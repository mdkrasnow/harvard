\documentclass[12pt,oneside]{article}
\usepackage[margin = 1in]{geometry}
\usepackage{graphicx}
\usepackage{color}
\usepackage{physics}
\usepackage{amsmath,amssymb,amsthm,mathtools}
\usepackage{fancyhdr}

\theoremstyle{definition}
\newtheorem{problem}{Problem}


\begin{document}
\pagestyle{fancy}
\fancyhead[L]{Math 112: Introductory Real Analysis}
\fancyhead[R]{Harvard University, Spring 2025}


\begin{center}
\bf \Large
Problem Set 1 \\[0.5 em]
\large
Due Wednesday, February 5, 2025
\end{center}
\begin{center}
\bf \Large
Matt Krasnow
\end{center}

\bigskip

\begin{problem}
(Exercise 1.1 in Rudin) 
If $r$ is a non-zero rational number and $x$ is an irrational number, prove that $r + x$ and $r x$ are irrational. 
\end{problem}


\textbf{Proof:} I will prove this by contradiction.

\section*{Proving that $r + x$ is irrational}  

\textbf{Step 1: Assume the contrary.}  
Assume, for contradiction, that $r + x$ is rational.  
By definition of rational numbers, we can write:

\[
r = \frac{m}{n}, \quad m, n \in \mathbb{Z}, \quad n \neq 0
\]

\[
r + x = \frac{a}{b}, \quad a, b \in \mathbb{Z}, \quad b \neq 0
\]

\textbf{Step 2: Solve for $x$.}  
Rearranging the equation:

\[
x = \frac{a}{b} - \frac{m}{n}
\]

\textbf{Step 3: Simplify the expression.}  
Since the rationals are closed under subtraction, we simplify:

\[
x = \frac{a n - m b}{b n}
\]

Since integers are closed under multiplication and subtraction, the numerator and denominator are integers.  
Thus, $x$ is of the form $\frac{c}{d}$ where $c, d \in \mathbb{Z}$ and $d \neq 0$, meaning $x$ is rational.

\textbf{Step 4: Contradiction.}  
This contradicts our original assumption that $x$ is irrational.  
Thus, our assumption that $r + x$ is rational must be false, and we conclude:

\[
r + x \text{ is irrational.}
\]

\section*{Proving that $r x$ is irrational}

\textbf{Step 1: Assume the contrary.}  
Assume, for contradiction, that $r x$ is rational.  
By definition of rational numbers, we can write:

\[
r x = \frac{c}{d}, \quad c, d \in \mathbb{Z}, \quad d \neq 0
\]

\textbf{Step 2: Express $x$ explicitly.}  
Since $r$ is rational, we substitute $r = \frac{m}{n}$:

\[
\frac{m}{n} x = \frac{c}{d}
\]

\textbf{Step 3: Solve for $x$.}  
Rearranging,

\[
x = \frac{c n}{m d}
\]

\textbf{Step 4: Check rationality.}  
Since integers are closed under multiplication, we see that $x$ is of the form:

\[
x = \frac{f}{g}, \quad f, g \in \mathbb{Z}, \quad g \neq 0
\]

which implies $x$ is rational.

\textbf{Step 5: Contradiction.}  
This contradicts our original assumption that $x$ is irrational.  
Thus, our assumption that $r x$ is rational must be false, and we conclude:

\[
r x \text{ is irrational.}
\]

\qed

\newpage

\begin{problem}
(Exercise 1.4 in Rudin) 
Let $E$ be a non-empty subset of an ordered set. 
Suppose $\alpha$ is a lower bound of $E$ and $\beta$ is an upper bound of $E$. 
Prove that $\alpha \leq \beta$. 
\end{problem}

\textbf{Proof.} I will prove this directly.

An ordered set is defined in Rudin 1.5 to have the following properties:

\begin{enumerate}
    \item If \( x, y \in S \), then exactly one of the following is true:
    \[
    x \leq y, \quad x \geq y, \quad x = y
    \]

    \item If \( x, y, z \in S \), if
    \[
    x < y \quad \text{and} \quad y < z
    \]
    then
    \[
    x < z.
    \]
\end{enumerate}

By definition of a lower bound,
\[
\forall x \in E, \quad \alpha \leq x.
\]

By definition of an upper bound,
\[
\forall x \in E, \quad \beta \geq x.
\]

Let \( x_i \) be the \( i \)th value in \( E \),
\[
\alpha \leq x_i \quad \text{and} \quad x_i \leq \beta.
\]

Thus, by the transitive property,
\[
\alpha \leq \beta.
\]

\(\square\)


\newpage 

\begin{problem}
(Exercise 1.5 in Rudin)
Let $A$ be a non-empty set of real numbers which is bounded below. 
Let $-A$ be the set of all numbers $-x$, where $x\in A$. 
Prove that 
\[
\inf A = - \sup (-A).
\]
\end{problem}
\textbf{Proof:} I will prove this directly.

By definition,

\[
\inf(A) \leq X \quad \forall X \in A
\]

\[
\sup(-A) \geq X \quad \forall X \in -A
\]

By negating a set, the order is reversed by definition.

Thus, 

the least number becomes the largest, the largest becomes the least, etc.

Thus, it follows that

\[
- \sup(-A) \leq X \quad \forall X \in -(-A)
\]

because \( \sup(-A) \) is the greatest value in the set \( A \).

By properties of fields, \( -(-X) = X \).

Thus, \( -(-A) = A \).

It follows that

\[
- \sup(-A) \leq X \quad \forall X \in A
\]

This is the same form as the \( \inf(A) \).

Thus,

\[
\inf(A) = -\sup(-A)
\]

\(\square\)

\newpage

\begin{problem}
Let $A \subset \mathbb{R}$ be a non-empty set of real numbers which is bounded above. 
Prove that, for any positive real number $\epsilon > 0$, there exists $x \in A$ such that $\sup A - x \leq \epsilon$. 
\end{problem}

\section*{Proof}
I will prove this directly using case work.

\subsection*{Case 1: \( \sup(A) \in A \)}
If \( \sup(A) \in A \), this implies that 
\[
\sup(A) - \sup(A) \leq \epsilon
\]
for any value \( \epsilon \in \mathbb{R}_{>0} \) because
\[
\sup(A) - \sup(A) = 0.
\]

\subsection*{Case 2: \( \sup(A) \notin A \)}
Since \( A \) is bounded from above and is a subset of the real numbers, the values in \( A \) must approach but never achieve \( \sup(A) \). 

This indicates that \( A \) is an infinitely long set which tends to \( \sup(A) \), getting infinitely closer in value.

Since there exist infinitely close values to \( \sup(A) \) which are in \( A \),
\[
\sup(A) - x \text{ must be less than any value of } \epsilon \in \mathbb{R}_{>0} \text{ for some value of } x.
\]



\end{document}
