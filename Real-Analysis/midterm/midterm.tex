\documentclass[12pt,oneside]{article}
\usepackage[margin=1in]{geometry}
\usepackage{graphicx}
\usepackage{color}
\usepackage{physics}
\usepackage{amsmath,amssymb,amsthm,mathtools}
\usepackage{fancyhdr}
\usepackage[most]{tcolorbox}

\theoremstyle{definition}
\newtheorem{problem}{Problem}

% Define a new tcolorbox environment for solution spaces
\newtcolorbox{solution}{
  breakable,
  enhanced,
  colback=teal!10,       % light teal background
  colframe=teal!70!black, % darker teal frame
  title=Solution
}

% Define a new tcolorbox environment for questions
\newtcolorbox{question}{
  breakable,
  enhanced,
  colback=blue!10,       % light blue background
  colframe=blue!70!black, % darker blue frame
  title=Question
}

\pagestyle{fancy}
\fancyhead[L]{Math 112: Introductary Real Analysis}
\fancyhead[R]{Harvard Universtiy, Spring 2025}

\begin{document}

\begin{center}
  \textbf{\Large Midterm}\\[0.5em]
  \large Spring 2025
\end{center}

\vspace{2em}

\noindent \textbf{Name: Matt Krasnow} 

\vspace{2em}

\noindent \textbf{Honor Pledge:}\\[0.5em]
On my honor, I declare that I have folowed the rules of this examination.

\vfill

\newpage

\begin{solution}
[Solution for Problem 1]:

\textbf{(a)} To prove:
\[
X\setminus A^\circ = \overline{(X\setminus A)}.
\]

Interior and closure are duel, so:
\[
A^\circ = X\setminus \overline{(X\setminus A)}.
\]

For \(x\in X\):
\begin{itemize}
    \item If \(x\in A^\circ\), then some open ball \(B_r(x)\subset A\). No open ball around \(x\) intersects \(X\setminus A\), so \(x\notin \overline{(X\setminus A)}\).
    \item If \(x\notin A^\circ\), every open ball around \(x\) intersects \(X\setminus A\), makeing \(x\) a limit point of \(X\setminus A\) (or in \(X\setminus A\)). So \(x\in \overline{(X\setminus A)}\).
\end{itemize}

Taking complements gives:
\[
X\setminus A^\circ = \overline{(X\setminus A)}.
\]







\textbf{(b)} prove:
\[
X\setminus \overline{A} = (X\setminus A)^\circ.
\]

For any \(x\in X\):
\begin{itemize}
    \item If \(x\in (X\setminus A)^\circ\), some open ball \(B_r(x)\subset X\setminus A\). This ball doesn't intersect \(A\), so \(x\) can't be a limit point of \(A\), meaning \(x\notin \overline{A}\). Thus \(x\in X\setminus \overline{A}\).
    \item If \(x\in X\setminus \overline{A}\), there's an open neighborhood of \(x\) not intersecting \(A\). So \(x\) is in the interior of \(X\setminus A\), i.e., \(x\in (X\setminus A)^\circ\).
\end{itemize}

So the equality holds:
\[
X\setminus \overline{A} = (X\setminus A)^\circ.
\]

\textbf{(c)} To prove:
\[
\delta A = \overline{A} \cap (X\setminus A^\circ).
\]

Boundry is defined as:
\[
\delta A = \overline{A}\setminus A^\circ.
\]

For sets \(B\) and \(C\), we know that:
\[
B\setminus C = B\cap (X\setminus C).
\]

Taking \(B=\overline{A}\) and \(C=A^\circ\):
\[
\delta A = \overline{A}\cap (X\setminus A^\circ).
\]
\end{solution}

\newpage











\begin{solution}
[Solution for Problem 2]:

Let \(X\) be metric spase, \(K\subset X\) compact, and \(O\subset X\) open with \(K\subset O\). 

For each \(x\in K\), since \(O\) is open and \(x\in O\), theres \(\varepsilon_x>0\) such that 
\[
B_{\varepsilon_x}(x)\subset O.
\]

The collection \(\{B_{\varepsilon_x/2}(x): x\in K\}\) forms an open cover of \(K\). By compactness, we get points \(x_1, x_2, \dots, x_n\in K\) where
\[
K\subset \bigcup_{i=1}^n B_{\varepsilon_{x_i}/2}(x_i).
\]

Define
\[
\varepsilon = \min\left\{\frac{\varepsilon_{x_1}}{2},\frac{\varepsilon_{x_2}}{2},\dots,\frac{\varepsilon_{x_n}}{2}\right\}.
\]

So \(\varepsilon > 0\). For any \(x\in K\), we have \(x\in B_{\varepsilon_{x_i}/2}(x_i)\) for some \(i\). Take any \(y\in B_\varepsilon(x)\). By triangle inequality:
\[
d(y,x_i) \le d(y,x) + d(x,x_i) < \varepsilon + \frac{\varepsilon_{x_i}}{2} \le \frac{\varepsilon_{x_i}}{2} + \frac{\varepsilon_{x_i}}{2} = \varepsilon_{x_i}.
\]

So \(y\in B_{\varepsilon_{x_i}}(x_i)\subset O\). Thus for every \(x\in K\),
\[
B_\varepsilon(x) \subset O.
\]
\end{solution}

\newpage












\begin{solution}
[Solution for Problem 3]:

\textbf{(a)} Let \(\{a_n\}\) and \(\{b_n\}\) be Cauchy sequences in \(X\). We'll show \(\{d(a_n,b_n)\}\) converges. 

For any \(m,n\in \mathbb{N}\), triangle inequality gves:
\[
d(a_n, b_n) \le d(a_n, a_m) + d(a_m, b_m) + d(b_m, b_n).
\]

Rearranging:
\[
\big|d(a_n, b_n) - d(a_m, b_m)\big| \le d(a_n,a_m) + d(b_n,b_m).
\]

Since \(\{a_n\}\) and \(\{b_n\}\) are Cauchy, for anny \(\varepsilon>0\) there's \(N\) where for all \(n,m\ge N\), both \(d(a_n,a_m)\) and \(d(b_n,b_m)\) are less than \(\varepsilon/2\). So
\[
\big|d(a_n, b_n) - d(a_m, b_m)\big| < \varepsilon.
\]

Thus \(\{d(a_n,b_n)\}\) is Cauchy in \(\mathbb{R}\), which is complete, so the sequence converges

\textbf{(b)} The relation 
\[
\{a_n\} \sim \{b_n\} \quad\text{if}\quad \lim_{n\to\infty}d(a_n,b_n)=0
\]
is an equivalance relation
\begin{itemize}
    \item \textbf{Reflexivity:} For all \(\{a_n\}\), \(d(a_n,a_n)=0\) always, so \(\lim_{n\to\infty} d(a_n,a_n)=0\). Hence, \(\{a_n\}\sim \{a_n\}\).
    \item \textbf{Symmetry:} If \(\{a_n\}\sim \{b_n\}\), then \(\lim_{n\to\infty}d(a_n,b_n)=0\). Since \(d(a_n,b_n)=d(b_n,a_n)\), we get \(\lim_{n\to\infty} d(b_n,a_n)=0\), so \(\{b_n\}\sim \{a_n\}\).
    \item \textbf{Transitivity:} If \(\{a_n\}\sim \{b_n\}\) and \(\{b_n\}\sim \{c_n\}\), than
    \[
    d(a_n,c_n) \le d(a_n,b_n) + d(b_n,c_n).
    \]
    Taking limits:
    \[
    \lim_{n\to\infty}d(a_n,c_n) \le \lim_{n\to\infty}d(a_n,b_n) + \lim_{n\to\infty}d(b_n,c_n)= 0 + 0 = 0.
    \]
    So \(\{a_n\}\sim \{c_n\}\).
\end{itemize}

\textbf{(c)} Define \(\Delta\) on equivalence classies by
\[
\Delta(A,B) := \lim_{n\to\infty}d(a_n,b_n),
\]
where \(\{a_n\}\in A\) and \(\{b_n\}\in B\) are representitives. 

To show its well-defined, take \(\{a_n\}\) and \(\{a'_n\}\) in \(A\) (so \(\lim_{n\to\infty}d(a_n,a'_n)=0\)), and \(\{b_n\}\) and \(\{b'_n\}\) in \(B\) (so \(\lim_{n\to\infty}d(b_n,b'_n)=0\)). By triangle inequality,
\[
\big|d(a_n,b_n)-d(a'_n,b'_n)\big|\le d(a_n,a'_n)+d(b_n,b'_n).
\]

Taking limitts:
\[
\lim_{n\to\infty}\big|d(a_n,b_n)-d(a'_n,b'_n)\big| \le \lim_{n\to\infty} d(a_n,a'_n)+\lim_{n\to\infty}d(b_n,b'_n)=0.
\]

So
\[
\lim_{n\to\infty}d(a_n,b_n)=\lim_{n\to\infty}d(a'_n,b'_n),
\]
makeing \(\Delta\) independant of representitives.

\textbf{(d)} \(\Delta\) is a metric on \(X^*\) (equiv classes of Cauchy sequences in \(X\)).
\begin{itemize}
    \item \textbf{Non-negativitty:} For any \(A,B\in X^*\),
    \[
    \Delta(A,B)=\lim_{n\to\infty}d(a_n,b_n)\ge0.
    \]
    \item \textbf{Identity:} \(\Delta(A,B)=0\) iff \(\lim_{n\to\infty}d(a_n,b_n)=0\), meaning \(\{a_n\}\sim \{b_n\}\), so \(A=B\).
    \item \textbf{Symmerty:} Since \(d(a_n,b_n)=d(b_n,a_n)\), we have \(\Delta(A,B)=\Delta(B,A)\).
    \item \textbf{Triangle Inequalty:} For classes \(A\), \(B\), \(C\) with representitives \(\{a_n\}\), \(\{b_n\}\), \(\{c_n\}\),
    \[
    d(a_n,c_n) \le d(a_n,b_n)+d(b_n,c_n).
    \]
    Taking limits:
    \[
    \Delta(A,C)=\lim_{n\to\infty}d(a_n,c_n) \le \lim_{n\to\infty}d(a_n,b_n)+\lim_{n\to\infty}d(b_n,c_n)=\Delta(A,B)+\Delta(B,C).
    \]
\end{itemize}
\end{solution}

\newpage







\begin{solution}
  [Solution for Problem 4 (Extra Credit)]:
  
  We'll prove $X^*$ (equiv classes of Cauchy sequences with metric $\Delta$) is complete.
  
  Let $\{A_k\}$ be Cauchy in $(X^*,\Delta)$. For eech $k$, pick a representative Cauchy sequence $\{a^{(k)}_n\}$ where $A_k$ is its class. Since $\{A_k\}$ is Cauchy, for anny $\varepsilon>0$ there's $N$ where for all $k,m\ge N$,
  $$
  \Delta(A_k,A_m)=\lim_{n\to\infty}d(a^{(k)}_n,a^{(m)}_n)<\varepsilon.
  $$
  
  This means for fixed $n$ (with large $k,m$), the values $d(a^{(k)}_n,a^{(m)}_n)$ are small. Using diagonal argument, construct $\{b_n\}$ in $X$ as the limit of the double sequence $\{a^{(k)}_n\}$: for each $n$, pick $b_n$ as the limit (or cluster point) of $\{a^{(k)}_n\}_{k}$.
  
  Then:
  \begin{enumerate}
     \item $\{b_n\}$ is Cauchy in $X$. This comes from combinng the Cauchy property of each $\{a^{(k)}_n\}$ with the uniform closness from the Cauchy condition on $\{A_k\}$.
     \item If $A$ is the class of $\{b_n\}$, then by triangle inequality and convergence properties,
     $$
     \Delta(A_k,A) \to 0 \quad \text{as } k\to\infty.
     $$
  \end{enumerate}
  
  So every Cauchy sequance in $X^*$ converges in $X^*$, making it complete
  \end{solution}
  
\end{document}
